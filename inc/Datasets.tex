\subsection*{Datasets}

\subsubsection*{ATIS (Air Travel Information System) Dataset}

A relational schema is used to organize data from the official airline guide in the ATIS corpus. There are 25 tables containing information about fares, airlines, flights, cities, airports, and ground services. All questions related to this dataset can be answered using a single relational query. The relational database uses shorter tables for this dataset to answer queries intuitively.

% Here is an example query from the ATIS dataset: Input is in natural language, and the output is in \lambda calculus.
% \begin{figure}[htb]
%     \centering
%     \includegraphics[width=0.8\textwidth]{pics/db/ATIS.png}
%     \caption{Example from ATIS dataset for semantic parsing}
%     \label{fig:ATIS}
% \end{figure}

\subsubsection*{GeoQuery Dataset}

United States geography is represented in the Geoquery dataset. About 800 facts are expressed in Prolog. State, city, river, and mountain information can be found in the database. Geographic and topographical attributes such as capitals and populations make up the majority of the attributes.

\subsubsection*{IMDb Dataset}

The IMDb dataset contains 50K reviews from IMDb. There is a limit of 30 reviews per movie\cite{maas-EtAl:2011:ACL-HLT2011}. Positive and negative reviews are equally represented in the dataset. The dataset creators considered a negative review with a score of 4 out of 10 and a positive review with a score of 7 out of 10. When creating the dataset, neural reviews are not taken into account. Furthermore, Training and testing datasets are equally divided.

\begin{figure}[htb]
    \centering
    \includegraphics[width=0.8\textwidth]{pics/db/IMDb.png}
    \caption{Database Structure of IMDb dataset}
    \label{fig:IMDb}
\end{figure}

\newpage % to avoid page break

\subsubsection*{Advising Dataset}

The Advising dataset was created in order to propose improvements in text2SQL systems. The creators of the dataset compare human-generated and automatically generated questions, citing properties of queries that relate to real-world applications. Dataset consists of questions from university students about courses that lead to particularly complex queries. The database contains fictional student records. The dataset includes student profile information, such as recommended courses, grades, and previous courses. In an academic advising meeting, students were asked to formulate questions they would ask if they knew the database. Many of the queries in this dataset were the same as those in ATIS, GeoQuery, and Scholar.

\begin{figure}[htb]
    \centering
    \includegraphics[width=0.5\textwidth]{pics/db/Advising.png}
    \caption{Example from Advising dataset \cite{vig_comparison_2019}}
    \label{fig:Advising}
\end{figure}

\subsubsection*{WikiSQL Dataset}

WikiSQL consists of 80K+ natural language questions and corresponding SQL queries on 24K+ tables extracted from Wikipedia. Neither the train nor development sets contain the database in the test set. Databases and SQL queries have simplified the dataset's creators' assumptions. This dataset consists only of SQL labels covering a single SELECT column and aggregation and WHERE conditions. Furthermore, all the databases contain only one table.

The database does not include complex queries involving advanced operations like JOIN, GROUP BY, ORDER BY, etc. Prior to the release of SPIDER, this dataset was considered to be a benchmark dataset. Using WikiSQL has been the subject of a great deal of research. WikiSQL's "WHERE" clause has been recognized as one of the most challenging clauses to parse semantically, and SQLNet and SyntaxSQL were previous state-of-the-art models.


\begin{figure}[htb]
    \centering
    \includegraphics[width=0.5\textwidth]{pics/db/WikiSQL.png}
    \caption{Example from WikiSQL dataset\cite{hwang_comprehensive_2019}}
    \label{fig:WikiSQL}
\end{figure}

% \newpage % to avoid page break

\subsubsection*{Spider Dataset}

Yale University students created this dataset.
The SPIDER database contains 10K questions and 5K+ complex SQL queries covering 138 different domains across 200 databases. As opposed to previous datasets (most of which used only one database), this one incorporates multiple datasets. Creating this corpus was primarily motivated by the desire to tackle complex queries and generalize across databases without requiring multiple interactions.

\begin{figure}[htb]
    \centering
    \includegraphics[width=0.6\textwidth]{pics/db/Spider.png}
    \caption{Example from Spider dataset\cite{yu_spider_2019}}
    \label{fig:Spider}
\end{figure}

Creating a dataset involves three main aspects: SQL pattern coverage, SQL consistency, and question clarity. Several databases from WikiSQL are included in the dataset. The table is complex as it links several tables with foreign keys. In SPIDER, SQL queries include: SELECT with multiple columns and aggregations, WHERE, GROUP BY, HAVING, ORDER BY, LIMIT, JOIN, INTERSECT, EXCEPT, UNION, NOT IN, OR, AND, EXISTS, LIKE.

\begin{figure}[htb]
    \centering
    \includegraphics[width=0.8\textwidth]{pics/db/Spider2.png}
    \caption{Example of Question-Query set from SPIDER\cite{yu_spider_2019}}
    \label{fig:Spider2}
\end{figure}

SPIDER's exact matching accuracy was 12.4\% compared to existing state-of-the-art models. As a result of its low accuracy, SPIDER presents a strong research challenge. Current SPIDER accuracy is around 75.5\% with an exact set match without values (refers to values in the WHERE clause) and around 72.6\% with values.