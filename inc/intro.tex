\section{Introduction}

Data retrieval in databases is typically done using SQL (Structured Query Language). Text-to-SQL machine learning models are a recent development in state-of-the-art research. The technique is an attractive alternative for many natural language problems, including complex queries and extraction tasks. The text is converted into a SQL query that can be executed on the database. This technique can save time and effort for both developers and end-users by enabling them to interact with databases through natural language queries. With the help of machine learning and knowledge-based resources, text language to SQL conversion is facilitated.

Semantic parsing is a natural language processing that extracts the meaning from text. Text-to-SQL, a type of Semantic Parsing, is a task that converts natural language problems into SQL query statements. 
This is achieved using machine learning and natural language processing algorithms, and this research is conducted to study different solutions and practices which has been taken by researchers to tackle this problem.

Text-to-SQL allows the elaboration of structured data with information about the natural language text in several domains, such as healthcare, customer service, and search engines. It can be used by data analysts, data scientists, software engineers, and end users who want to explore and analyze their data without learning SQL.
It can be used in a variety of ways:

1) Data analysts can use it to generate SQL queries for specific business questions, such as "What are the top ten products sold this month?"

2) Data scientists can use it to generate SQL queries for machine learning experiments, such as "How does the price of these products affect their sales?"

3) Businesses can use this technique to automate data extraction and improve efficiency.

4) End-users who want to explore and analyze their data without learning SQL can use it by clicking on a button on any table or chart in a user interface.

Although these models may not solve this problem entirely and perfectly, humans can still struggle with the task. For example, people involved in database migration projects often have to work on schema that they have never seen before.

This research study will review some of the most commonly used NLP technologies relevant to converting text language into Structured Query Language (SQL), and representative models and datasets in the recent solutions for this challenge and their technical implementation.
